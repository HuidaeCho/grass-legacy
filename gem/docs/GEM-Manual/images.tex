\batchmode

\documentclass[11pt,english]{article}
\RequirePackage{ifthen}


\usepackage{pslatex}
\usepackage[T1]{fontenc}
\usepackage[latin1]{inputenc}
\usepackage{geometry}
\geometry{verbose,a4paper,tmargin=2.5cm,bmargin=2.5cm,lmargin=2cm,rmargin=2cm}
\setcounter{secnumdepth}{2}
\setcounter{tocdepth}{2}

\setlength \parskip{\medskipamount}

\setlength \parindent{0pt}
\usepackage{calc}
\usepackage{color}
\usepackage{setspace}
\onehalfspacing
\IfFileExists{url.sty}{\usepackage{url}}
                      {%
\providecommand{\url}{\texttt} }


\makeatletter

%
\newenvironment{lyxcode}{\begin{list}{}{
\setlength{\rightmargin}{\leftmargin}
\setlength{\listparindent}{0pt}% needed for AMS classes
\raggedright
\setlength{\itemsep}{0pt}
\setlength{\parsep}{0pt}
\normalfont\ttfamily}%
 \item[]}
{\end{list}} 


\usepackage{babel}
\makeatother



\pagecolor[gray]{.7}

\usepackage[]{inputenc}



\makeatletter

\makeatletter
\count@=\the\catcode`\_ \catcode`\_=8 
\newenvironment{tex2html_wrap}{}{}%
\catcode`\<=12\catcode`\_=\count@
\newcommand{\providedcommand}[1]{\expandafter\providecommand\csname #1\endcsname}%
\newcommand{\renewedcommand}[1]{\expandafter\providecommand\csname #1\endcsname{}%
  \expandafter\renewcommand\csname #1\endcsname}%
\newcommand{\newedenvironment}[1]{\newenvironment{#1}{}{}\renewenvironment{#1}}%
\let\newedcommand\renewedcommand
\let\renewedenvironment\newedenvironment
\makeatother
\let\mathon=$
\let\mathoff=$
\ifx\AtBeginDocument\undefined \newcommand{\AtBeginDocument}[1]{}\fi
\newbox\sizebox
\setlength{\hoffset}{0pt}\setlength{\voffset}{0pt}
\addtolength{\textheight}{\footskip}\setlength{\footskip}{0pt}
\addtolength{\textheight}{\topmargin}\setlength{\topmargin}{0pt}
\addtolength{\textheight}{\headheight}\setlength{\headheight}{0pt}
\addtolength{\textheight}{\headsep}\setlength{\headsep}{0pt}
\setlength{\textwidth}{349pt}
\newwrite\lthtmlwrite
\makeatletter
\let\realnormalsize=\normalsize
\global\topskip=2sp
\def\preveqno{}\let\real@float=\@float \let\realend@float=\end@float
\def\@float{\let\@savefreelist\@freelist\real@float}
\def\liih@math{\ifmmode$\else\bad@math\fi}
\def\end@float{\realend@float\global\let\@freelist\@savefreelist}
\let\real@dbflt=\@dbflt \let\end@dblfloat=\end@float
\let\@largefloatcheck=\relax
\let\if@boxedmulticols=\iftrue
\def\@dbflt{\let\@savefreelist\@freelist\real@dbflt}
\def\adjustnormalsize{\def\normalsize{\mathsurround=0pt \realnormalsize
 \parindent=0pt\abovedisplayskip=0pt\belowdisplayskip=0pt}%
 \def\phantompar{\csname par\endcsname}\normalsize}%
\def\lthtmltypeout#1{{\let\protect\string \immediate\write\lthtmlwrite{#1}}}%
\newcommand\lthtmlhboxmathA{\adjustnormalsize\setbox\sizebox=\hbox\bgroup\kern.05em }%
\newcommand\lthtmlhboxmathB{\adjustnormalsize\setbox\sizebox=\hbox to\hsize\bgroup\hfill }%
\newcommand\lthtmlvboxmathA{\adjustnormalsize\setbox\sizebox=\vbox\bgroup %
 \let\ifinner=\iffalse \let\)\liih@math }%
\newcommand\lthtmlboxmathZ{\@next\next\@currlist{}{\def\next{\voidb@x}}%
 \expandafter\box\next\egroup}%
\newcommand\lthtmlmathtype[1]{\gdef\lthtmlmathenv{#1}}%
\newcommand\lthtmllogmath{\dimen0\ht\sizebox \advance\dimen0\dp\sizebox
  \ifdim\dimen0>.95\vsize
   \lthtmltypeout{%
*** image for \lthtmlmathenv\space is too tall at \the\dimen0, reducing to .95 vsize ***}%
   \ht\sizebox.95\vsize \dp\sizebox\z@ \fi
  \lthtmltypeout{l2hSize %
:\lthtmlmathenv:\the\ht\sizebox::\the\dp\sizebox::\the\wd\sizebox.\preveqno}}%
\newcommand\lthtmlfigureA[1]{\let\@savefreelist\@freelist
       \lthtmlmathtype{#1}\lthtmlvboxmathA}%
\newcommand\lthtmlpictureA{\bgroup\catcode`\_=8 \lthtmlpictureB}%
\newcommand\lthtmlpictureB[1]{\lthtmlmathtype{#1}\egroup
       \let\@savefreelist\@freelist \lthtmlhboxmathB}%
\newcommand\lthtmlpictureZ[1]{\hfill\lthtmlfigureZ}%
\newcommand\lthtmlfigureZ{\lthtmlboxmathZ\lthtmllogmath\copy\sizebox
       \global\let\@freelist\@savefreelist}%
\newcommand\lthtmldisplayA{\bgroup\catcode`\_=8 \lthtmldisplayAi}%
\newcommand\lthtmldisplayAi[1]{\lthtmlmathtype{#1}\egroup\lthtmlvboxmathA}%
\newcommand\lthtmldisplayB[1]{\edef\preveqno{(\theequation)}%
  \lthtmldisplayA{#1}\let\@eqnnum\relax}%
\newcommand\lthtmldisplayZ{\lthtmlboxmathZ\lthtmllogmath\lthtmlsetmath}%
\newcommand\lthtmlinlinemathA{\bgroup\catcode`\_=8 \lthtmlinlinemathB}
\newcommand\lthtmlinlinemathB[1]{\lthtmlmathtype{#1}\egroup\lthtmlhboxmathA
  \vrule height1.5ex width0pt }%
\newcommand\lthtmlinlineA{\bgroup\catcode`\_=8 \lthtmlinlineB}%
\newcommand\lthtmlinlineB[1]{\lthtmlmathtype{#1}\egroup\lthtmlhboxmathA}%
\newcommand\lthtmlinlineZ{\egroup\expandafter\ifdim\dp\sizebox>0pt %
  \expandafter\centerinlinemath\fi\lthtmllogmath\lthtmlsetinline}
\newcommand\lthtmlinlinemathZ{\egroup\expandafter\ifdim\dp\sizebox>0pt %
  \expandafter\centerinlinemath\fi\lthtmllogmath\lthtmlsetmath}
\newcommand\lthtmlindisplaymathZ{\egroup %
  \centerinlinemath\lthtmllogmath\lthtmlsetmath}
\def\lthtmlsetinline{\hbox{\vrule width.1em \vtop{\vbox{%
  \kern.1em\copy\sizebox}\ifdim\dp\sizebox>0pt\kern.1em\else\kern.3pt\fi
  \ifdim\hsize>\wd\sizebox \hrule depth1pt\fi}}}
\def\lthtmlsetmath{\hbox{\vrule width.1em\kern-.05em\vtop{\vbox{%
  \kern.1em\kern0.8 pt\hbox{\hglue.17em\copy\sizebox\hglue0.8 pt}}\kern.3pt%
  \ifdim\dp\sizebox>0pt\kern.1em\fi \kern0.8 pt%
  \ifdim\hsize>\wd\sizebox \hrule depth1pt\fi}}}
\def\centerinlinemath{%
  \dimen1=\ifdim\ht\sizebox<\dp\sizebox \dp\sizebox\else\ht\sizebox\fi
  \advance\dimen1by.5pt \vrule width0pt height\dimen1 depth\dimen1 
 \dp\sizebox=\dimen1\ht\sizebox=\dimen1\relax}

\def\lthtmlcheckvsize{\ifdim\ht\sizebox<\vsize 
  \ifdim\wd\sizebox<\hsize\expandafter\hfill\fi \expandafter\vfill
  \else\expandafter\vss\fi}%
\providecommand{\selectlanguage}[1]{}%
\makeatletter \tracingstats = 1 


\begin{document}
\pagestyle{empty}\thispagestyle{empty}\lthtmltypeout{}%
\lthtmltypeout{latex2htmlLength hsize=\the\hsize}\lthtmltypeout{}%
\lthtmltypeout{latex2htmlLength vsize=\the\vsize}\lthtmltypeout{}%
\lthtmltypeout{latex2htmlLength hoffset=\the\hoffset}\lthtmltypeout{}%
\lthtmltypeout{latex2htmlLength voffset=\the\voffset}\lthtmltypeout{}%
\lthtmltypeout{latex2htmlLength topmargin=\the\topmargin}\lthtmltypeout{}%
\lthtmltypeout{latex2htmlLength topskip=\the\topskip}\lthtmltypeout{}%
\lthtmltypeout{latex2htmlLength headheight=\the\headheight}\lthtmltypeout{}%
\lthtmltypeout{latex2htmlLength headsep=\the\headsep}\lthtmltypeout{}%
\lthtmltypeout{latex2htmlLength parskip=\the\parskip}\lthtmltypeout{}%
\lthtmltypeout{latex2htmlLength oddsidemargin=\the\oddsidemargin}\lthtmltypeout{}%
\makeatletter
\if@twoside\lthtmltypeout{latex2htmlLength evensidemargin=\the\evensidemargin}%
\else\lthtmltypeout{latex2htmlLength evensidemargin=\the\oddsidemargin}\fi%
\lthtmltypeout{}%
\makeatother
\setcounter{page}{1}
\onecolumn

% !!! IMAGES START HERE !!!

\setcounter{secnumdepth}{2}
\setcounter{tocdepth}{2}
\stepcounter{section}
\stepcounter{section}
\stepcounter{section}
\stepcounter{subsection}
\stepcounter{subsection}
\stepcounter{subsection}
{\newpage\clearpage
\lthtmlpictureA{tex2html_wrap1211}%
\framebox{\parbox[t][1\totalheight]{1\columnwidth}{%
Note: please be aware that GEM can only install modules in a specially
prepared package. See section 4 for how to do this.%
}}%
%
\lthtmlpictureZ
\lthtmlcheckvsize\clearpage}

\stepcounter{subsection}
{\newpage\clearpage
\lthtmlpictureA{tex2html_wrap1235}%
\framebox{\parbox[t][1\totalheight]{1\columnwidth}{%
Note: Operation of GEM under Cygwin is only possible in verbose mode
(option {}``-v'' or {}``--verbose'')! Please read section A.2
on Windows specific issues.%
}}%
%
\lthtmlpictureZ
\lthtmlcheckvsize\clearpage}

{\newpage\clearpage
\lthtmlpictureA{tex2html_wrap1243}%
\framebox{\begin{minipage}[t][1\totalheight]{1\columnwidth}%
Note: in order to install extensions that do not provide binaries,
you need to have a C compiler (preferably GNU CC) and make tools installed.
Please refer to the OS specific section A for more details.
\par
For unpacking extension packages you need the appropriate software.
Depending on the format of the archive, this could be \emph{tar},
\emph{gzip}, \emph{bunzip2} and \emph{unzip}.
\par
If you want to get extensions from an internet source (http or ftp),
you also need \emph{wget}. These programs will very likely already
be installed on your system. If not, you will find it easy to locate
a copy for your OS using an internet search engine.%
\end{minipage}}%
%
\lthtmlpictureZ
\lthtmlcheckvsize\clearpage}

{\newpage\clearpage
\lthtmlpictureA{tex2html_wrap1289}%
\framebox{\begin{minipage}[t][1\totalheight]{1\columnwidth}%
Note: an extension may also provide shell scripts that do not need
to be compiled. Installation does, however, by default work in the
same way. Look for the keyword {}``scripts'' in the extension details
(under {}``Binary installation files'') by querying it as discussed
in section 3.5 on installation of binaries. If it exists, you can
perform a binary install (again, see section 3.5) for {}``scripts'',
skipping configuration and compilation steps and avoiding the need
to have C development tools installed. %
\end{minipage}}%
%
\lthtmlpictureZ
\lthtmlcheckvsize\clearpage}

{\newpage\clearpage
\lthtmlpictureA{tex2html_wrap1363}%
\framebox{\begin{minipage}[t][1\totalheight]{1\columnwidth}%
Note: Awful detail: you must use the short action name {}``-q''
in this case, not {}``--query='' and it has to be at the end of
the GEM command line!%
\end{minipage}}%
%
\lthtmlpictureZ
\lthtmlcheckvsize\clearpage}

\stepcounter{subsection}
\stepcounter{subsection}
\stepcounter{subsection}
\stepcounter{section}
{\newpage\clearpage
\lthtmlpictureA{tex2html_wrap1519}%
\framebox{\begin{minipage}[t][1\totalheight]{1\columnwidth}%
Note: if your extension needs to install {}``unusual'' things (additional
fonts, tcl widgets, ...) you may need to adapt the top-level Makefile
in your extension package.%
\end{minipage}}%
%
\lthtmlpictureZ
\lthtmlcheckvsize\clearpage}

\stepcounter{subsection}
{\newpage\clearpage
\lthtmlpictureA{tex2html_wrap1531}%
\framebox{\begin{minipage}[t][1\totalheight]{1\columnwidth}%
Note: the {}``--test='' action simulates the entire installation
process, including checks for dependencies and already installed extensions.
This means that testing may result in an error message after successful
compilation. If this annoys, you, use the {}``--force'' option.%
\end{minipage}}%
%
\lthtmlpictureZ
\lthtmlcheckvsize\clearpage}

\stepcounter{subsection}
{\newpage\clearpage
\lthtmlpictureA{tex2html_wrap1547}%
\framebox{\begin{minipage}[t][1\totalheight]{1\columnwidth}%
Note: please always provide licensing information!%
\end{minipage}}%
%
\lthtmlpictureZ
\lthtmlcheckvsize\clearpage}

\stepcounter{subsection}
\stepcounter{subsection}
{\newpage\clearpage
\lthtmlpictureA{tex2html_wrap1569}%
\framebox{\begin{minipage}[t][1\totalheight]{1\columnwidth}%
Note: all other things inside {}``<'' and {}``>'' will be filtered
out as HTML tags, even if they are not!%
\end{minipage}}%
%
\lthtmlpictureZ
\lthtmlcheckvsize\clearpage}

{\newpage\clearpage
\lthtmlpictureA{tex2html_wrap1575}%
\framebox{\begin{minipage}[t][1\totalheight]{1\columnwidth}%
Note: your modules' individual HTML manual pages have to be slightly
adapted: please prefix \texttt{<a href=>} style references to GRASS
modules that are not part of the extension with \texttt{../../html/}. %
\end{minipage}}%
%
\lthtmlpictureZ
\lthtmlcheckvsize\clearpage}

\stepcounter{subsection}
\stepcounter{subsection}
{\newpage\clearpage
\lthtmlpictureA{tex2html_wrap1603}%
\framebox{\begin{minipage}[t][1\totalheight]{1\columnwidth}%
Note: be very careful when adapting the \emph{uninstall} script! It
is run with superuser privileges (or those of whatever user owns the
GRASS installation directory). If you provide wrong paths there is
no limit to the damage it can do to the user's system! Try to keep
your changes to a minimum if you have to make any.%
\end{minipage}}%
%
\lthtmlpictureZ
\lthtmlcheckvsize\clearpage}

{\newpage\clearpage
\lthtmlpictureA{tex2html_wrap1609}%
\framebox{\begin{minipage}[t][1\totalheight]{1\columnwidth}%
Note: be very careful when adapting the \emph{post} script! It is
run with superuser privileges (or those of whatever user owns the
GRASS installation directory). If you add flawed commands there is
no limit to the damage it can do to the user's system!%
\end{minipage}}%
%
\lthtmlpictureZ
\lthtmlcheckvsize\clearpage}

\stepcounter{subsection}
\stepcounter{subsection}
{\newpage\clearpage
\lthtmlpictureA{tex2html_wrap1627}%
\framebox{\begin{minipage}[t][1\totalheight]{1\columnwidth}%
Note: due to great heterogenity of installed system libraries, it
is probably not worth the effort trying to create generic linux binaries.
Binaries will work better for homogeneous platforms such as Cygwin
and Lorenzo Moretti's GRASS for Mac OS X.%
\end{minipage}}%
%
\lthtmlpictureZ
\lthtmlcheckvsize\clearpage}

{\newpage\clearpage
\lthtmlpictureA{tex2html_wrap1725}%
\framebox{\begin{minipage}[t][1\totalheight]{1\columnwidth}%
Note: if your extension consists of only platform independent things
that need not be compiled (shell script, tcl code, ...), you can create
a set of binaries on any system in the same way as described above
and call the binary set \char`\"{}scripts\char`\"{} this will allow
the user to install those scripts without having developer tools installed!%
\end{minipage}}%
%
\lthtmlpictureZ
\lthtmlcheckvsize\clearpage}

\stepcounter{subsection}
\stepcounter{subsection}
\stepcounter{subsubsection}
\stepcounter{subsubsection}
\stepcounter{subsubsection}
\stepcounter{subsubsection}
\stepcounter{subsubsection}
\appendix
\stepcounter{section}
\stepcounter{subsection}
{\newpage\clearpage
\lthtmlpictureA{tex2html_wrap1807}%
\framebox{\begin{minipage}[t][1\totalheight]{1\columnwidth}%
Note: if you want to be able to use any extension package, not only
those that provide binaries for Mac OS X, you need to install the
complete GNU C development tools from the Mac OS X installation media.%
\end{minipage}}%
%
\lthtmlpictureZ
\lthtmlcheckvsize\clearpage}

\stepcounter{subsection}
{\newpage\clearpage
\lthtmlpictureA{tex2html_wrap1825}%
\framebox{\parbox[t][1\totalheight]{1\columnwidth}{%
Note: Operation of GEM under Cygwin is only possible in verbose mode
(option {}``-v'' or {}``--verbose'')! The problem is that Windows
cannot redirect output to stderr. This interferes with GEM's message
hiding.%
}}%
%
\lthtmlpictureZ
\lthtmlcheckvsize\clearpage}

\stepcounter{subsubsection}
\stepcounter{section}
\stepcounter{subsection}
\stepcounter{subsubsection}
\stepcounter{subsubsection}
\stepcounter{subsection}
\stepcounter{subsubsection}
\stepcounter{subsubsection}
\stepcounter{subsubsection}
\stepcounter{subsubsection}
\stepcounter{subsubsection}
\stepcounter{subsubsection}
\stepcounter{subsubsection}
\stepcounter{subsection}
\stepcounter{subsection}

\end{document}
