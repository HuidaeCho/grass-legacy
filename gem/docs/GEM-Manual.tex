%% LyX 1.4.0 created this file.  For more info, see http://www.lyx.org/.
%% Do not edit unless you really know what you are doing.
\documentclass[11pt,english]{article}
\usepackage{pslatex}
\usepackage[T1]{fontenc}
\usepackage[latin1]{inputenc}
\usepackage{geometry}
\geometry{verbose,a4paper,tmargin=2.5cm,bmargin=2.5cm,lmargin=2cm,rmargin=2cm}
\setcounter{secnumdepth}{2}
\setcounter{tocdepth}{2}
\setlength\parskip{\medskipamount}
\setlength\parindent{0pt}
\usepackage{calc}
\usepackage{color}
\usepackage{setspace}
\onehalfspacing
\IfFileExists{url.sty}{\usepackage{url}}
                      {\newcommand{\url}{\texttt}}

\makeatletter
%%%%%%%%%%%%%%%%%%%%%%%%%%%%%% Textclass specific LaTeX commands.
\newenvironment{lyxcode}
{\begin{list}{}{
\setlength{\rightmargin}{\leftmargin}
\setlength{\listparindent}{0pt}% needed for AMS classes
\raggedright
\setlength{\itemsep}{0pt}
\setlength{\parsep}{0pt}
\normalfont\ttfamily}%
 \item[]}
{\end{list}}

%%%%%%%%%%%%%%%%%%%%%%%%%%%%%% User specified LaTeX commands.


\usepackage{babel}
\makeatother
\begin{document}

\title{GRASS Extensions Manager (GEM), Version 1.0\\
Manual}


\author{Benjamin Ducke (benducke@compuserve.de)}

\maketitle
\begin{abstract}
GRASS 6 Extensions Manager (GEM) is a small stand-alone program intended
to make it easy for GRASS GIS users to download, compile and install
additional GRASS modules. GEM manages source code, scripts and pre-compiled
binaries in a simple file layout called an \emph{extension}. Extensions
are accompanied by a set of ASCII files that store all relevant information
including dependencies on other extensions or specific GRASS versions.
Extensions can be stored conveniently in a single compressed archive
file, a so-called \emph{extension package} using external programs
such as tar and gzip. An extension (package) can be created easily
by copying existing source codes into the right places of a skeleton
file layout and filling in some information in simple, commented ASCII
files. Makefiles written for GRASS 6 should work with minimal changes
as GEM uses a simplified version of the orginal GRASS make system.

This Document provides instructions for using GEM and for writing
portable GRASS extensions to be installed with GEM.

\newpage{}
\end{abstract}
\tableofcontents{}

\newpage{}


\section{Why Do I Need GEM?}

GRASS GIS (\url{www.grass.itc.it}) is a powerful, modularized open
source Geographic Information System. The base distribution comes
with hundreds of useful modules. However, there are a few flaws in
the design:

\begin{enumerate}
\item Installation of add-on modules that are not part of the main distribution
(\url{http://grass.gdf-hannover.de/twiki/bin/view/GRASS/GrassAddOns})
is tedious. It requires a full copy of the GRASS source codes. Add-on
sources must be copied to the right locations, compilation and installation
must then be started as though one was to install the whole system
from scratch.
\item Users willing to install add-on functionality must have at least a
basic knowledge of how to compile and install modules from C source
code.
\item The sheer number of modules makes it hard to find the one that has
exactly the functionality needed. There is no thematic grouping of
modules. This becomes worse as more add-on modules are installed.
\item Developers of add-on functionality currently have no way to make this
process more user-friendly. Add-on developers as a rule write modules
that are functionaly closely related. These should be grouped and
distributed as a package along with some over-arching documentation
to bind them together and make them more accessible from a user's
perspective.
\item Some add-ons modules exist in the official CVS and in another place
such as the developer's homepage. For a user, it is hard to know which
version is more current. In fact, GRASS has no versioning scheme except
for the base distribution itself.
\end{enumerate}
GEM, the GRASS Extensions Manager, was developed as an open source
solution to address all of these issues. It is a tool that simplifies
development, distribution and installation of additional modules (extensions)
for GRASS GIS (version 6.0 and above).

From the GRASS user's perspective GEM can be used to:

\begin{itemize}
\item Conveniently download and install add-on GRASS modules (extensions)
that are not part of the CVS.
\item Avoid having to keep a complete GRASS source tree on the disk for
installing new functionality.
\item Install add-on modules from provided binaries without the need to
have any development tools installed.
\item Manage installed extensions: query, update, uninstall them.
\end{itemize}
Form the GRASS developer's perspective GEM allows for:

\begin{enumerate}
\item Development of GRASS modules completely outside the CVS tree. In a
dedicated directory that is much easier to maintain and sync.
\item Packaging of module source code, documentation and pre-compiled binaries
and deployment as a single file (extension package).
\item Simple outsourcing of sets of GRASS functionality from the base distribution
into extension packages.
\end{enumerate}
GEM works on Linux/Unix, MacOS X and cygwin systems. There are some
OS specific issues which are discussed in the appendix.


\section{How Does GEM Work?}

The GEM program itself is a relatively simple frontend written in
ANSI C that interacts with more complex installation scripts (make
system) contained in each individual extension package. These scripts
are a scaled-down and slightly modified version of the original GRASS
6 make system. Files in an extension package correspond 1:1 with the
layout of the original GRASS source tree, the only difference being
that an extension package only contains source code for the add-on
modules plus a few things that are needed to setup the source code,
parse HTML documentation etc.

A GRASS extension is essentially a minimal replication of the GRASS
source tree including all necessary makefiles. It contains only the
source code for a few modules that constitute the extensions. On the
top level, you will find a number of ASCII files that contain the
information gem needs to manage the extension. All information is
managed in plain ASCII format. 

Installed extensions will be registered in \emph{\$(PATH\_TO\_GRASS)/etc/extensions.db}.
This registry file also contains version and dependencies information.
Each extension should also provide an uninstall script to be run when
the user wants to get rid of that extension. 

They are stored in \emph{\$(PATH\_TO\_GRASS)/etc/uninstall.<extension\_name>}. 

Extension may add a submenu to the GIS Manager. The first extension
installed creates an additional \char`\"{}Xtns\char`\"{} top level
menu under which each extension can register a nested submenu. 

Gem modifies \$\emph{(PATH\_TO\_GRASS)/etc/d.m/menu.tcl} for this. 

The extension source code is in the \emph{src} directory, along with
all the necessary makefiles. Other directories may hold pre-compiled
binaries and should be named appropriately (\emph{win32}, \emph{macosx},
...). 

Extensions may provide code for C language modules and libraries,
scripts and HTML documentation. 

GEM compiles and install add-on modules using the sources and makefiles
contained in the extension.

For the user, GEM is really simple to use. All that is required is
a download of the extension package to install and knowledge of where
the local GRASS installation resides. If the user does not have permission
to install new files in the GRASS installation directory, a password
for a user with sufficient permissions will be required (see next
section).

Developers of add-on modules will (hopefully) also find the process
of migrating their work to a GEM extension package a matter of minutes
and well worth the effort. More information for developers can be
found in section \ref{sec:Developing-Extensions-for} of this document.


\section{Installation and Usage}

Note: in addition to GEM itself, you will probably want to install
GNU C compiler and maketools for compilation. Tar, gzip, unzip, bunzip2
for handling various archive formats. Wget for fetching extensions
from the internet. 


\subsection{Program Installation\label{sub:Program-Installation}}

You can download GEM source code and binaries for several different
platforms from the author's homepage (\url{http://www.uni-kiel.de/ufg/index1.htm}).
Check the links at the bottom of the page. You will also find several
extension packages for installation with GEM.

Download the file \emph{gem-someversion.tar.gz} to a convenient location,
unpack it and change into the newly created directory:

\begin{lyxcode}
tar~-xzvf~gem-\emph{someversion}.tar.gz

cd~gem-\emph{someversion}
\end{lyxcode}
In this directory, you will find a folder \emph{bin} that contains
binaries for different platforms. Change into it and list its contents:

\begin{lyxcode}
cd~bin

ls
\end{lyxcode}
You will see a number of programs. Pick the one that corresponds to
your OS and start it. E.g., if you are working in a cygwin environment
under Windows, do:

\begin{lyxcode}
./gem-cygwin.exe
\end{lyxcode}
You should get a text explaining the program's options on your screeen.

If you want (and have the required permissions), you can copy the
file to a directory for system-wide binaries so you can call GEM from
anywhere without having to prefix a directory path (./ in the example
above) by typing just \texttt{gem}:

\begin{lyxcode}
cp~gem-cygwin.exe~/usr/local/bin/gem
\end{lyxcode}
Once GEM has reached a stable release version (1.0), it will also
become part of the GRASS CVS version. If you download and install
such a version on your computer, you will automatically have GEM available
system-wide (usually \emph{/usr/local/bin}) and can just start it
with:

\begin{lyxcode}
gem
\end{lyxcode}
Instructions on how to user GEM to install an actual extension are
given in section \ref{sub:Basic-Usage}.


\subsection{Installation from Source Code}

%
\begin{minipage}[t][1\totalheight]{1\columnwidth}%
Note: please make sure that you system-wide linker path includes the
location of the GRASS dynamic libraries (e.g. \emph{/usr/local/grass-6.1.cvs/libs}).
This is usually done by adding an appropriate entry to \emph{/etc/ld.so.conf}
and running \emph{ldconfig}. You may have to consult your system's
administrator.%
\end{minipage}%


You must have a C compiler, preferably GNU C, and the corresponding
make tools installed. Most Linux/Unix systems should come with these
installed. If not, use your distributions package manager to install
them (look for something like {}``Development tools''). Mac OS X
and Cygwin users: see OS specific notes in section \ref{sec:OS-Specific-Issues}.

Download the file \emph{gem-someversion.tar.gz} to a convenient location,
unpack it and change into the newly created directory:

\begin{lyxcode}
tar~-xzvf~gem-\emph{someversion}.tar.gz

cd~gem-\emph{someversion}
\end{lyxcode}
In this directory, start the compilation process:

\begin{lyxcode}
make
\end{lyxcode}
Afer a few seconds, the compilation is done and you have an executable
file that you can start from the current directory:

\begin{lyxcode}
./gem
\end{lyxcode}
\ldots{} or copy the executable to a system-wide directory, such
as \emph{/usr/local/bin}.


\subsection{Quickstart}

%
\framebox{\parbox[t][1\totalheight]{1\columnwidth}{%
Note: please be aware that GEM can only install modules in a specially
prepared package. See section 4 for how to do this.%
}}%


For the impatient. To install an extension into your running version
of GRASS: start a GRASS session, download an extension and install
it from inside the GRASS session using:

\begin{lyxcode}
gem~-{}-install=\emph{ExtensionName}.tar.gz
\end{lyxcode}
Provide the superuser password if needed. Restart GIS Manager and
look in the {}``Xtns'' menu for new modules (some modules may not
provide such entries). If anything goes wrong or confuses you: read
the rest of this section!


\subsection{Basic Usage\label{sub:Basic-Usage}}

%
\framebox{\parbox[t][1\totalheight]{1\columnwidth}{%
Note: Operation of GEM under Cygwin is only possible in verbose mode
(option {}``-v'' or {}``--verbose'')! Please read section A.2
on Windows specific issues.%
}}%


This section will show you how to use GEM to perform basic things:
installing, querying and removing (un-installing) extensions for GRASS
GIS. For the sake of simplicity, I will assume that your GRASS installation
resides in \emph{/usr/local/grass-6.x.y} and you may have to adjust
this path in the examples below, so that it reflects your individual
setup (Mac OS X users: see notes on system specific issues in section
\ref{sub:Mac-OS-X} for how to find the path to your installation).
Also, the extension package used in the following examples is called
\emph{RasterTools.tar.gz} and is in {}``tar'd gzip'' format. Replace
this with the name of the extension you wish to install as needed.

I will further assume that you (or your system's administrator) have
copied the executable \emph{gem} into a directory from where it can
be started system-wide (such as \emph{/usr/local/bin}) without having
to prefix the path to the executable (see instructions in \ref{sub:Program-Installation}).

%
\framebox{\begin{minipage}[t][1\totalheight]{1\columnwidth}%
Note: in order to install extensions that do not provide binaries,
you need to have a C compiler (preferably GNU CC) and make tools installed.
Please refer to the OS specific section A for more details.

For unpacking extension packages you need the appropriate software.
Depending on the format of the archive, this could be \emph{tar},
\emph{gzip}, \emph{bunzip2} and \emph{unzip}.

If you want to get extensions from an internet source (http or ftp),
you also need \emph{wget}. These programs will very likely already
be installed on your system. If not, you will find it easy to locate
a copy for your OS using an internet search engine.%
\end{minipage}}%


GEM understands short and long options. Long options are just a more
legible version of the short options, which is why I will use them
in this document. E.g. {}``-i'' is a short option that does exactly
the same as {}``--install=''. To see all options, short and long,
simply call GEM without any options or {}``--help'':

\begin{lyxcode}
gem
\end{lyxcode}
The {}``--version'' action shows information about the GEM version
you are using.

You will notice that GEM knows a special sort of options called {}``actions''
this are options that cause GEM to operate in some way on the extension.
The regular {}``options'' are just used to modify the way GEM works.

To install a module into your GRASS installation (in this example
\emph{/usr/local/grass-6.x.y}), simply pass the name of the archive
containing the extension or the directory with the unpacked files
to the {}``--install='' action and supply the path to the GRASS
installation for which you wish to install the extension (using the
{}``--grass='' option): 

\begin{lyxcode}
gem~-{}-grass=/usr/local/grass-6.x.y~-{}-install=RasterTools.tar.gz
\end{lyxcode}
If everything goes well, you will see a few messages on the screen
and after a short while the program will inform you that it is done.
If you do not have permission to write into the system-wide GRASS
installation directory, you will be asked for the password of the
user owning the GRASS binary files. 

If any errors occur, use option {}``-v'' ({}``--verbose'') to
see what is going on. 

By default, the {}``--install='' action performs an installation
from source code. This will first configure the sources for your system's
individual setup, then compile the source code into binaries, finally
install them in the appropriate locations.

%
\framebox{\begin{minipage}[t][1\totalheight]{1\columnwidth}%
Note: an extension may also provide shell scripts that do not need
to be compiled. Installation does, however, by default work in the
same way. Look for the keyword {}``scripts'' in the extension details
(under {}``Binary installation files'') by querying it as discussed
in section 3.5 on installation of binaries. If it exists, you can
perform a binary install (again, see section 3.5) for {}``scripts'',
skipping configuration and compilation steps and avoiding the need
to have C development tools installed. %
\end{minipage}}%


If you invoke GEM from inside a running GRASS session, you can ommit
the {}``--grass='' option. GEM will then automatically install the
extension into the GRASS installation that is currently running. We
will assume this to be true for the following usage examples, as it
saves me some typing work (\ldots{}). 

The new modules provided by the extension should be available to you
immediately from within a GRASS session. The extension will be registered
as {}``RasterTools''. To get a list of newly installed GRASS modules:

\begin{lyxcode}
gem~-{}-query=RasterTools
\end{lyxcode}
This will display all sorts of information about your freshly installed
extension, including the installed modules (under {}``Commands provided'').
If it is too much information to display on your terminal, pipe it
through \emph{more} and press space to see one page after another:

\begin{lyxcode}
gem~-{}-query=RasterTools~|~more
\end{lyxcode}
Alternatively, you can browse the GRASS HTML offline help which should
now contain a link to the extension and its modules from its main
page \emph{index.html}. 

Some extensions may provide menu entries for a GRASS GUI. Currently,
\emph{d.m} and \emph{gis.m} are supported by GEM. Both of these GUIs
need to be restarted. If the extension provides menu entries, you
will find them under {}``Xtns'' in the main menu bar.

If you want to know details about an extension before you install
it, you can also query the extension package itself:

\begin{lyxcode}
gem~-{}-query=RasterTools.tar.gz
\end{lyxcode}
Actions {}``--details='' and {}``--license='' will give additional
information if needed. To list all installed extensions (shows name,
version and type of installation, i.e. binaries or compiled from source)
in the current GRASS installation, just query without any extension
name: 

\begin{lyxcode}
gem~-q
\end{lyxcode}
%
\framebox{\begin{minipage}[t][1\totalheight]{1\columnwidth}%
Note: Awful detail: you must use the short action name {}``-q''
in this case, not {}``--query='' and it has to be at the end of
the GEM command line!%
\end{minipage}}%


You can also un-install an extension to remove it from the system-wide
GRASS installation (take name from list produced by above command).
Again, you need write-access to the GRASS installation dir.

\begin{lyxcode}
gem~-{}-uninstall=AdvancedViewshedAnalysis~
\end{lyxcode}
This should clean your system of all installed modules, HTML pages
and GUI menu entires. Quit and restart the GIS-Manager (\emph{d.m})
to see the effects of these actions. If something goes wrong or you
overwrite GIS Manager's menu config file (\emph{\$(PATH\_TO\_GRASS)/\-etc/\-dm/\-menu.tcl}),
or you have installed a new version of GRASS (e.g. from CVS) and are
now missing GUI menu entries and HTML pages for your extensions:

\begin{lyxcode}
gem~-{}-restore~
\end{lyxcode}
\ldots{} will try to put everything back in order. The program will
also create a backup copy \emph{menu.tcl.gem.bak} before it alters
\emph{menu.tcl}. The \char`\"{}--restore\char`\"{} action is also
useful if you update or re-install GIS-Manager or the GRASS HTML-Documentation
in which case both menu entries and HTML references have to be restored
for all extensions installed. 


\subsection{Installation of Pre-compiled Binary Extension Files\label{sub:Installation-of-Pre-compiled}}

Some extensions may provide pre-compiled binary files for one or more
operating systems. A user wishing to install an extension that provides
binaries for his OS does not need to have C development tools installed.
This may frequently be the case for users of Mac OS X and Cygwin (but
see OS specific issues in section \ref{sec:OS-Specific-Issues}).

It is recommended for all users of GRASS and GEM to install C development
tools and install extensions from sources (the regular way as described
in section \ref{sub:Basic-Usage}). This will compile the source code
into custom binaries that are optimally tailored to your system. However,
there may be circumstances that make using pre-compiled binaries appear
more convenient or even unavoidable.

You can check whether an extension provides such pre-compiled binary
files for your OS by querying the extension package:

\begin{lyxcode}
gem~-{}-query=RasterTools.tar.gz
\end{lyxcode}
If any binaries are provided, they will be listed under \char`\"{}Binary
installation files\char`\"{}. You should be able to see whether they
are suitable for your system from the names. E.g. {}``cygwin'' would
provide binaries for Cygwin users (see section \ref{sub:Developers'-Guidelines}
for conventional names of binaries for different OS).

Use the option \char`\"{}--binary=\char`\"{} in conjunction with the
\char`\"{}--install=\char`\"{} action to install the binaries you
deem right:

\begin{lyxcode}
gem~-{}-binary=cygwin~-{}-install=RasterTools.tar.gz
\end{lyxcode}
This process should finish quicker than regular installation from
source code as it will skip source code configuration and compilation.

If you have chosen a wrong set of binaries, new commandes will simply
fail to start when you try to use them.


\subsection{Additional Options}

Option {}``-f'' ({}``--force'') can be used to force GEM to re-install
an existing extension, over-writing anything that was installed previously.
This is not a recommended thing to do! There is currently no clean
updating mechanism for GEM, so you are advised to first de-install
the existing extension (which might involve de-installing all dependent
extensions first), then install freshly.

You can also download and install, query etc. an extension packages
directly from an internet source (http or ftp), provided that \emph{wget}
is installed (\url{http://www.gnu.org/software/wget/wget.html}):

\begin{lyxcode}
gem~-{}-install=http://www.uni-kiel.de/ufg/dateienDucke/RasterTools.tar.gz
\end{lyxcode}

\subsection{Module Versions and Dependencies}

Information about all installed extensions is stored in the file \emph{etc/extensions.db}
in your system-wide GRASS installation directory. This file contains
the names, versions and \emph{dependencies} of all installed extensions.
Some extensions may need another extension or a particular GRASS version
to be installed before it can function properly. If an extension's
dependencies are not met, GEM will abort the installation and you
need to first install all required software.

Querying an extension package will let you see the dependencies it
has before you attempt to install (under {}``Dependencies''):

\begin{lyxcode}
gem~-{}-query=RasterTools.tar.gz
\end{lyxcode}
Compare this with the output of

\begin{lyxcode}
gem~-q
\end{lyxcode}
... and you will know if you need to install something else first.

You can force installation of extensions with unmatched dependencies
by using the \char`\"{}--force\char`\"{} option. This is not a recommended
thing to do! 

GEM also guards against de-installation of extensions that are still
needed by other extensions still present on the system. You need to
un-install all dependent extension in reverse order of installation
first.

There is currently no clean updating mechanism for GEM, so you are
advised to first un-install an existing extension (which might involve
de-installing all dependent extensions first) before you install a
newer version.

Do not edit \emph{extensions.db} manually, unless you now \emph{exactly}
what you are doing!


\section{Developing Extensions for Use with GEM\label{sec:Developing-Extensions-for}}

Converting existing GRASS add-on modules to a GEM extension package
is easy. Extensions may provide C program code and headers for modules
and libraries, shell scripts and tcl code. Basically anything that
you find a directory for in the skeleton package.

%
\framebox{\begin{minipage}[t][1\totalheight]{1\columnwidth}%
Note: if your extension needs to install {}``unusual'' things (additional
fonts, tcl widgets, ...) you may need to adapt the top-level Makefile
in your extension package.%
\end{minipage}}%



\subsection{GEM Developers' Support\label{sub:GEM-Developers'-Support}}

The skeleton package (see section \ref{sub:The-Skeleton-Package})
is an almost complete GRASS extension. Use this as a template for
starting new extensions or migrating existing source code from the
GRASS source tree. All you need to do is copy your source files, scripts
etc. into the appropriate places in the \emph{src} directory (you
will find that everything mirrors the way GRASS sources are organized)
and fill necessary information into the toplevel ASCII files.

GEM can configure, compile and install extensions from a plain directory.
Just provide the name of the directory for all actions. This allows
you to conveniently keep all your sources and documentations in a
small, portable directory outside the GRASS source tree and maintain
everything in there.

You can test whether an extension compiles and installs on your system
by using the {}``--test='' action. This will perform all steps except
for actually copying the files to their destinations.

%
\framebox{\begin{minipage}[t][1\totalheight]{1\columnwidth}%
Note: the {}``--test='' action simulates the entire installation
process, including checks for dependencies and already installed extensions.
This means that testing may result in an error message after successful
compilation. If this annoys, you, use the {}``--force'' option.%
\end{minipage}}%


The {}``--clean='' action merely exists for the convenience of the
developer: it performs a make clean in the extensions \emph{src} directory.

Two more options exist to make life easier for developers: {}``--skip-config''
to skip source code configuration for speeding up the testing. and
{}``--options='' (sorry about the awful name). Everything you specify
here will be passed through to the C compiler upon compilation of
your extension.


\subsection{The Skeleton Package\label{sub:The-Skeleton-Package}}

The skeleton package is an almost complete GRASS extension. Use this
as a template for starting new extensions or migrating existing source
code from the GRASS source tree. All you need to do is copy your source
files, scripts etc. into the appropriate places in the \emph{src}
directory (you will find that everything mirrors the way GRASS sources
are organized) and fill necessary information into the toplevel ASCII
files (see following sections).

Look into the skeleton extension directory and open the ASCII files
with a texteditor of your choice. You will find lots of comments that
help you make sense of their contents. You can put comments starting
with \char`\"{}\#\char`\"{} at the beginning of a line or at the end
in any file. These will be filtered out upon parsing of the file by
GEM.

The skeleton contains a copy of the GPL as the default license. Creators
of new extensions need to be aware of this! Either insert the name
of your extension at the end of that license or provide your own custom
licensing information.

%
\framebox{\begin{minipage}[t][1\totalheight]{1\columnwidth}%
Note: please always provide licensing information!%
\end{minipage}}%



\subsection{Arranging the Source Code\label{sub:Arranging-the-Source}}

You will find that your makefiles for individual moduls and libs can
stay largely the way they are. One thing that may need to be adjusted
is the \texttt{MODULE\_TOPDIR = ..} statement that points to the location
of the global GRASS makefile \emph{include} directory. Also be aware
that references to other include files libraries etc. that you reference
in a makefile may have to be adjusted depending on how you decide
to structure the source for your extension.

Take a look at the example module in \emph{src/raster/r.example}.


\subsection{Documenting Extension Files}

It is extremely important to provide sufficient documentation about
your extension's intended use, functionality, dependencies and shortcomings!

A number of ASCII files in the toplevel extension directory store
all this information about an extension. They must be edited appropriately.
GEM's {}``--query='', {}``--license='' and {}``--details=''
actions will dump their contents to the screen. These are:

\emph{authors}, \emph{bugs}, \emph{commands}, \emph{depends}, \emph{description},
\emph{entries-gisman}, \emph{entries-gisman2}, \emph{headers}, \emph{id},
\emph{info}, \emph{libs}, \emph{license} and \emph{name} and \emph{version}.

Files \emph{authors}, \emph{bugs}, \emph{commands}, \emph{description},
\emph{headers}, \emph{id}, \emph{info}, \emph{libs}: These files are
merely for the user's information. Their contents will be dumped by
the {}``--query='' action. Much of this information will also go
into the system-wide GRASS HTML help. Each extension registers its
own section in \emph{\$GISBASE/docs/html/index.html} which links to
an individual index page for that extension. This allows the user
convenient access to inidividual modules' help pages as well as the
contents of files \emph{description} and \emph{info}.

Because of this, some description files can contain HTML tags. These
will be ignored when the user queries the extension on a console but
will be interpreted when the same information is accessed from the
GRASS HTML offline help. There are two important exceptions: \texttt{<p>}
and \texttt{<br>} will produce paragraph and line breaks for console
output, as well. 

%
\framebox{\begin{minipage}[t][1\totalheight]{1\columnwidth}%
Note: all other things inside {}``<'' and {}``>'' will be filtered
out as HTML tags, even if they are not!%
\end{minipage}}%


Take a look at the files in the skeleton package for their individual
meanings and format. Files that will be parsed for HTML tags are:
\emph{authors}, \emph{bugs}, \emph{description} and \emph{info}.

%
\framebox{\begin{minipage}[t][1\totalheight]{1\columnwidth}%
Note: your modules' individual HTML manual pages have to be slightly
adapted: please prefix \texttt{<a href=>} style references to GRASS
modules that are not part of the extension with \texttt{../../html/}. %
\end{minipage}}%


File \emph{license}: this will be display by the {}``--license=''
action.

File \emph{info}: this will be displayed by the {}``--details=''
action and also integrated in the HTML index page for the extension.

Files \emph{depends}, \emph{name} and \emph{version}: see section
\ref{sub:Version-Information}.

Files \emph{entries-gisman} and \emph{entries-gisman2}: see section
\ref{sub:Providing-GUI-Hooks}.


\subsection{Version Information\label{sub:Version-Information}}

The file \emph{name} is a very crucial one! It contains the name under
which GEM will register your extension. Please see \ref{sub:Developers'-Guidelines}
about naming conventions!

The file \emph{version} stores the current version number of your
extension. Make sure to keep this up-to-date prior to new releases.

If your extension depends on other extensions or a specific GRASS
version, you can state this in the \emph{depends} file. See the example
in the skeleton extension for details. GEM will respect this information
when a user tries to install or uninstall any extension.


\subsection{Un-install and Post-install Actions}

Two files remain that have not been discussed yet: \emph{uninstall}
and \emph{post}.

The \emph{uninstall} file is a shell script that contains all commands
necessary to clean up the GRASS installation after the user invoked
GEM with the {}``--uninstall='' action to un-install an extension.
It takes care of deleting module binaries, HTML manpages, C include
files and libraries provided by that extension from \emph{\$GISBASE}.
In the most simple case, all you need to do is provide the list of
your extension's user commands in \texttt{EXT\_MODULES=''''}. For
more complex extensions, you may have to provide additional files
to delete or even customize \emph{uninstall}. Take a look at the file
in the skeleton extension to see what it does in detail.

GEM will copy your extension's uninstall script to \emph{\$GISBASE/etc/uninstall.ExtensionName}
and it will be run from there by the {}``--uninstall='' action.

%
\framebox{\begin{minipage}[t][1\totalheight]{1\columnwidth}%
Note: be very careful when adapting the \emph{uninstall} script! It
is run with superuser privileges (or those of whatever user owns the
GRASS installation directory). If you provide wrong paths there is
no limit to the damage it can do to the user's system! Try to keep
your changes to a minimum if you have to make any.%
\end{minipage}}%


The \emph{post} script can be used to customise many actions in case
your extension needs anything not provided by the regular make system
or GEM description files. GEM exports a number of environment variables
depending on what options and actions the user chose to run it with.
The \emph{post} script is automatically run after many actions and
can be used to carry out custom tasks depending on the type of action.
Take a look at the file provided by the skeletion extension to see
what can be done.

%
\framebox{\begin{minipage}[t][1\totalheight]{1\columnwidth}%
Note: be very careful when adapting the \emph{post} script! It is
run with superuser privileges (or those of whatever user owns the
GRASS installation directory). If you add flawed commands there is
no limit to the damage it can do to the user's system!%
\end{minipage}}%



\subsection{Providing GUI Hooks\label{sub:Providing-GUI-Hooks}}

GEM currently supports the installation of menu items for GIS Manager:
the old version (\emph{d.m}) and the new one (\emph{gis.m}) are both
support. Support for \emph{d.m} menus is quite limited. You can only
create one submenu and any number of menu items and separators inside
of it. Menus for \emph{gis.m} can contain any tk menu code. You can
have as many levels of menu hierarchy as you like. See the examples
in the skeleton extension to learn how they are organized.

The file responsible for creating \emph{d.m} menu entries is \emph{entries-gisman}.
For \emph{gis.m} it is \emph{entries-gisman2}. For \emph{d.m} menus,
\emph{entries-gisman} is directly merged into \emph{\$GISBASE/etc/dm/menu.tcl}.
GEM will also place a copy of the original \emph{entries-gisman} file
into \emph{\$GISBASE/etc/dm/gem-entries}. This is used by the {}``--restore''
action to restore \emph{d.m} menus in case the user updates the GRASS
installation and menu.tcl gets overwritten. For the purpose of un-installation,
markers are stored as comments at the end of \emph{menu.tcl}. This
allows the {}``--uninstall='' action to find an extensions menu
entries and delete them. The file \emph{menu.tcl} will be backed up
as \emph{menu.tcl.gem.bak} so a user can restore it if anything should
wrong.

Things work much simpler for \emph{gis.m}. In this case, \emph{\$GISBASE/etc/gm/gmmenu.tcl}
dynamically re-builds the {}``Xtns'' menu upon start-up. If \emph{entries-gisman2}
exists in an extension package, GEM will make sure to create an directory
\emph{Xtns} in \emph{\$GISBASE/etc/gm/} and copies \emph{entries-gisman2}
into it, renaming it to the extension's name. From these files, \emph{gis.m}
will automatically build the {}``Xtns'' menu via some parsing code
in \emph{gmmenu.tcl}.

Your submenu will be sorted in under the \char`\"{}Xtns\char`\"{}
menu in alphabetical position according to either the name of the
top level menu item (\emph{d.m}) or the extension name (\emph{gis.m})
it is smart to keep both the same (see guidelines in section \ref{sub:Developers'-Guidelines}). 

Both \emph{d.m} and \emph{gis.m} need to be restarted to see the effects
of installing new menu entries.


\subsection{Providing Binary Distributions}

%
\framebox{\begin{minipage}[t][1\totalheight]{1\columnwidth}%
Note: due to great heterogenity of installed system libraries, it
is probably not worth the effort trying to create generic linux binaries.
Binaries will work better for homogeneous platforms such as Cygwin
and Lorenzo Moretti's GRASS for Mac OS X.%
\end{minipage}}%


It is possible to create an extension package that will

If not done yet, unpack the extension files into a directory:

\begin{lyxcode}
tar~-xzvf~extensionname.tar.gz
\end{lyxcode}
Compile the extension under the OS that you wish to prepare binaries
for (see OS specific issues in section \ref{sec:OS-Specific-Issues}
about the additional software you may need to do this). You do not
need to install, compilation will suffice (don't worry if you get
an error message about the extension being already installed):

\begin{lyxcode}
./gem~-{}-grass=pathToGRASS~-t~extensionName
\end{lyxcode}
If the compilation was successful, you will find two new folders in
the \emph{src} directory of the extension directory: \emph{bin.architecture-osname}
and \emph{dist.architecture-osname}. E.g. for a Windows/Cygwin compilation
the {}``dist'' directory will frequently be

\begin{lyxcode}
dist.i686-pc-cygwin
\end{lyxcode}
(or similar). 

Create a new directory on the same level as the \emph{src} directory.
For an appropriate name, follow the guidelines in section \ref{sub:Developers'-Guidelines}.
In our Cygwin example, this would be \emph{cygwin}. In this directory:

\begin{enumerate}
\item move the \emph{dist.architecture-osname} directory here
\item copy \emph{src/Makefile} here
\item copy \emph{src/include} (with all subdirectories) here
\end{enumerate}
Make a slight modification to the copy of Makefile: at the top of
the file following the definitions for install directories (BINDIR=,
INST\_DIR=). Add another line that reads:

\begin{lyxcode}
GISBASE~=~\emph{dist.architecture-osname}
\end{lyxcode}
For our example, this might be:

\begin{lyxcode}
GISBASE~=~dist.i686-pc-cygwin
\end{lyxcode}
(or similar). This makes sure that the install command will install
binaries from the right directory.

Now, since we are preparing a binary distribution for the sake of
people who do not have any development tools installed, we need to
supply them with a copy of the GNU make tools, since this is the tool
that will take care of the actual installation. What we need is really
just the program \emph{make}. We will copy it into a folder \emph{bin}
under our new directory. In our example, then the copy would go into
\emph{cygwin/bin}.

The skeleton extension package already has copies of \emph{make} for
Cygwin and Mac OS X in place. If you need them for another architecture,
download sources from \url{http://www.gnu.org/software/make/} and
compile the appropriate binaries.

%
\framebox{\begin{minipage}[t][1\totalheight]{1\columnwidth}%
Note: if your extension consists of only platform independent things
that need not be compiled (shell script, tcl code, ...), you can create
a set of binaries on any system in the same way as described above
and call the binary set \char`\"{}scripts\char`\"{} this will allow
the user to install those scripts without having developer tools installed!%
\end{minipage}}%



\subsection{Preparing an Extension for Release}

1. make sure all ASCII files in the top-level extension directory
have the necessary information (licensing!).

2. Compile binaries that you may wish to provide for those poor people
who cannot afford C development tools

3. Clean the sources of compiled binaries (this will not affect binaries
that you wish to provide and copied into their individual directories).

4. If not done yet, delete the example module directory \emph{src/raster/r.example}.
And remove it from the list of subdirs in \emph{src/raster/Makefile}.

5. Pack everything into an archive using GNU tar and gzip.

6. Announce your shiny new extension on the appropriate GRASS mailling
list!

SHORTCOMINGS:

GEM is not very well-suited to install add-ons consisting of just
a single module. This would need an entire extension with the whole
shebang for a single module! I is assumed, that people who program
for a GIS as a rule produce more than just one module for a certain
purose and that these can well be grouped into extensions. 


\subsection{Developers' Guidelines\label{sub:Developers'-Guidelines}}


\subsubsection{Naming Extensions}

Please do not use anything fancy for your extension's name (as stored
in the \emph{name} file in the toplevel directory of your extension):
no special characters, no whitespaces (including simple {}``space'').
Use the same restrictions any sane programmer would use for file naming.
This ensures that GEM can always correctly parse your extension's
name.


\subsubsection{Packaging Extensions}

The recommended way to package an extension is to use GNU \emph{tar}
and \emph{gzip} utilities as these will normally be available on any
OS running GRASS. I recommend you use \emph{.tar.gz} as file extension:

\begin{lyxcode}
tar~-czvf~ExtensionName.tar.gz~DirectoryWithExtensionFiles
\end{lyxcode}
Make sure that you do not tar up the files from inside the extension
directory! This will result in a non-functioning package and also
gets users annoyed because the files will get decompressed directly
into whatever directory the extension was compiled and it will be
hard to clean up the mess. Also, do not use an absolute path to the
files. Rather, build the archive from the directory that contains
your extension directory, as shown in the example above.


\subsubsection{HTML-Documentation}

Check the GRASS HTML documentation and follow its style! Please provide
good documentation for your individual modules as well as some brief
and detailed information about the intended use, functionality and
shortcomings of your extension (files \emph{description} and \emph{info},
respectively).


\subsubsection{Binaries}

Suggested folder names for binary distributions:

Mac OS X: \emph{macosx}

Cygwin: \emph{cygwin}

Linux (glibc 2.2): \emph{linux22}

Linux (glibc 2.3): \emph{linux23}

Shell scripts: \emph{scripts}


\subsubsection{Menu Entries}

Name the Toplevel menu item in your menu files (\emph{entries-gisman}
and \emph{entries-gisman2}) the same as the Extension. This will make
things much clearer for the user!

\appendix

\section{OS Specific Issues\label{sec:OS-Specific-Issues}}


\subsection{Mac OS X\label{sub:Mac-OS-X}}

If you want to run GRASS under Mac OS X you can install Lorenzo Moretti's
binaries. They are frequently updated and easy to install. GEM has
been tested for this version of GRASS and works well with it. 

In order to get this working, you need to install some additinal software
from the Mac OS X install media. Just download from \url{http://wwwamb.bologna.enea.it/forgrass/}
and follow the instructions in the documentation that comes with the
program files.

%
\framebox{\begin{minipage}[t][1\totalheight]{1\columnwidth}%
Note: if you want to be able to use any extension package, not only
those that provide binaries for Mac OS X, you need to install the
complete GNU C development tools from the Mac OS X installation media.%
\end{minipage}}%


If you are using Lorenzo's binaries, the path to GRASS for any GEM
operation will be:

\emph{/Applications/Grass/grass6x.app/Contents/Resources/grass-6.x.y}

or

\emph{/Applications/Grass/grass61cvs.app/Contents/Resources/grass-6.1.cvs}

if you decided to also install Lorenzo's copy of the CVS version.

I cannot provide any information about compiling GRASS from sources
for Mac OS X as I do not have access to such a machine.


\subsection{GRASS under Windows with Cygwin\label{sub:GRASS-under-Windows}}

%
\framebox{\parbox[t][1\totalheight]{1\columnwidth}{%
Note: Operation of GEM under Cygwin is only possible in verbose mode
(option {}``-v'' or {}``--verbose'')! The problem is that Windows
cannot redirect output to stderr. This interferes with GEM's message
hiding.%
}}%


Although there is now work underway to create a native Win32 version
of GRASS, for now the only way to get it running is to use the Cygwin
emulation layer. For instructions on how to do this, see \url{http://geni.ath.cx/grass.html}.

Some additional hints about setting up Cygwin:

\begin{itemize}
\item Make sure you set the {}``Default Text File Type'' to {}``Unix''
during Setup.
\item If you want to be able to use any extension package, not only those
that provide binaries for Cygwin, jsut install everything in the {}``Devel''
category and you will have a complete development system.
\item If you want to be able install extension packages directly from the
internet (http or ftp sources) using GEM, make sure to install wget,
located the {}``Network'' category.
\item If you want to use an installation in a network environment with NT
domain authentication, you may need to update \emph{cygwin1.dll} to
a newer version. Download from \url{http://cygwin.com/snapshots/}
and replace the old version in your system search path.
\end{itemize}
Under Cygwin, there is no support for the Unix \emph{su} command.
This means that installation of extensions has to be done by someone
with appropriate access rights to the GRASS installation directory
inside the Cygwin installation directory (e.g. if you installed cygwin
to \emph{C:\textbackslash{}cygwin}, this would be something like \emph{C:\textbackslash{}cygwin\textbackslash{}usr\textbackslash{}local\textbackslash{}grass-someversion}).


\subsubsection{CygwinGRASS}

CygwinGRASS (\url{https://www.geographie.uni-freiburg.de/~mlechner/CygwinGRASS/})
is nice because it has all you need to install and use GRASS under
Windows on one CD. Unfortunately, it has not been updated for a while
and now contains fairly outdated versions of GRASS. See the document
on the CD for installation instructions. 

Some hints regarding the setup:

\begin{itemize}
\item The CygwinGRASS setup has a little flaw in that it assumes the wrong
location for the local package directory. Click on {}``Browse''
and select the folder repository on the setup medium.
\item If you want to be able to use any extension package, not only those
that provide binaries for Cygwin, jsut install everything in the {}``Devel''
category and you will have a complete development system.
\item If you want to be able install extension packages directly from the
internet (http or ftp sources) using GEM, make sure to install wget,
located the {}``Network'' category.
\item If you want to use a CygwinGRASS installation in a network environment
with NT domain authentication, you will need to update \emph{cygwin1.dll}
to a newer version. Download from \url{http://cygwin.com/snapshots/}
and replace the old version in your system search path.
\end{itemize}
Browsing HTML helpfiles with the default command {}``iexplore''
seems to fail. However, you can install a different browser and change
the environment variable \texttt{GRASS\_HTML\_BROWSER} accordingly.


\section{Technical Information}


\subsection{File Layout of a GRASS Extension}

Blue items are \texttt{\textcolor{blue}{directories}}, green ones
are \texttt{\textcolor{green}{files}}.


\subsubsection{Toplevel}

\texttt{\textcolor{green}{README}}

\texttt{\textcolor{green}{authors, bugs, commands, depends, description,
entries-gisman, entries-gisman2, headers, id, info, libs, license,
name, post, uninstall, version}}

\texttt{\textcolor{blue}{src {[}, cygwin, macosx, other-architecture,
\ldots{}]}}


\subsubsection{In Directory \emph{src}}

\texttt{\textcolor{green}{COPYING}}

\texttt{\textcolor{green}{README}}

\texttt{\textcolor{green}{REQUIREMENTS.HTML}}

\texttt{\textcolor{green}{Makefile, config.guess, config.sub, onfigure,
configure.in, install-sh}}

\texttt{\textcolor{blue}{db, demolocation, display, general, imagery,
include, lib, man, paint, ps, raster, raster3d, scripts, sites, tools,
vector, visualization}}


\subsection{Changes to the GRASS Make System Files}

GEM exports three environment variables that point to the directories
storing GRASS 6 headers and libs and to the install location (e.g.
\emph{/usr/local/grass-6.1.cvs}):

\texttt{GINSTALL\_INC}

\texttt{GINSTALL\_LIB}

\texttt{GINSTALL\_DST }

These have been added to the makefiles so that externally compiled
modules are able to see those GRASS headers and libs they need. Thus,
the following files were slightly altered:


\subsubsection{configure script}

Everything that checks for optional system libraries and could cause
configure to fail went out. This means that extensions that have to
check for optional system libraries will have to re-add these checks.

In summary print-out section at end of script: got rid of: \texttt{echo
\char`\"{} Startup script in directory: \$\{bindir\}\char`\"{}}. And
added some custom text that refers to the extension having been configured.


\subsubsection{main Makefile}

L28: \texttt{INST\_DIR} is set according to \texttt{GINSTALL\_DST}
which is exported by GEM.

L168: install target \texttt{real-install}: commented out everything
that seems not strictly necessary for a module installation and might
interfere with the user's GRASS installation. Now exits with error
code {}``1'' on write permission problems. This can be caught by
GEM and dealt with. Only creates \emph{error.log} if a module did
not compile. This is used to check for compilation errors and abort,
if necessary.

L171: commented out stuff for creating GRASS startup script


\subsubsection{include/Make/Grass.make.in}

L32: added \texttt{-I\$(GINSTALL\_INC)} to \texttt{GRASS\_INC} 

L53: added \texttt{-L\$(GINSTALL\_LIB)} to \texttt{ARCH\_LIBPATH} 

L81: added \texttt{\$(GEM\_C\_OPTS)} to \texttt{CFLAGS} 


\subsubsection{include/Make/Rules.make}

L37: the path to the HTML docs is now expanded with \emph{/extensions/<extensionName>.}
The name for the extension is read from \texttt{GEM\_EXT\_NAME} which
is exported by GEM 

L55: added \texttt{\$(GINSTALL\_LIB)} to definition of \texttt{LD\_LIBRARY\_PATH\_VAR}
.

L49: this now makes two different copies: one to go into the extension's
HTML folder \emph{docs/\-extensions/\-\$(GEM\_EXT\_NAME)} and one
-- with adjusted relative links -- to go into \emph{docs/html}. \texttt{GEM\_EXT\_NAME}
is exported by the GEM installer tool 


\subsubsection{include/Shlib.make}

L3: \texttt{SHLIB = \$(ARCH\_LIBDIR)/\$(SHLIB\_PREFIX)\$(SHLIB\_NAME).\$(GEM\_EXT\_VERSION)\$(SHLIB\_SUFFIX)} 

L14: \texttt{\$(SHLIB): \$(SHLIB\_OBJS) \$(SHLIB\_LD) -o \$@ \$(LDFLAGS)
\$\textasciicircum{} \$(EXTRA\_LIBS) \&\& ln -sf \$(notdir \$@) \$(patsubst
\%.\$(GEM\_EXT\_VERSION)\$(SHLIB\_SUFFIX),\%\$(SHLIB\_SUFFIX),\$@)} 

\texttt{GEM\_EXT\_VERSION} installs lib with version number as given
in \emph{version} file in extension package.


\subsubsection{lib/init/Makefile}

L32: commented out everything that seems not strictly necessary for
a module installation and might interfere with the user's GRASS installation.


\subsubsection{build\_html\_index.sh }

L16: expand HTML with \emph{/extensions/\$GEM\_EXT\_NAME} (exported
by GEM). The HTML default text has been altered in many places to
reflect its new function for describing extensions. GEM also exports
the following variables to enrich the HTML default text: \texttt{GEM\_EXT\_VERSION}
(extension's version number).

\newpage{}


\subsection{GEM Synopsis}

\begin{lyxcode}
Usage:~gem~{[}OPTION]~{[}ACTION]~{[}FILE|DIR]



Possible~ACTIONs~are:

~~-i,~-{}-install~~~~~~~~~install~a~GRASS~extension

~~-u,~-{}-uninstall~~~~~~~remove~an~extension~from~GRASS

~~-q,~-{}-query~~~~~~~~~~~display~information~about~extension/list~installed

~~-d,~-{}-details~~~~~~~~~display~additional~details~about~an~extension

~~-c,~-{}-clean~~~~~~~~~~~clean~extension's~source~code~directories

~~-t,~-{}-test~~~~~~~~~~~~configure~and~compile~extension,~but~don't~install

~~-l,~-{}-license~~~~~~~~~show~copyright~information~for~an~extension

~~-r,~-{}-restore~~~~~~~~~recreate~HTML~links~and~GIS~Manager~entries

~~-h,~-{}-help~~~~~~~~~~~~display~this~help~and~exit

~~-V,~-{}-version~~~~~~~~~output~version~information~and~exit



Possible~OPTIONs~are:

~~-g,~-{}-grass=PATH~~~~~~path~to~GRASS~installation~dir

~~-b,~-{}-binary=NAME~~~~~no~compilation:~use~binary~files~for~system~NAME

~~-f,~-{}-force~~~~~~~~~~~force~action,~regardless~of~dependencies

~~-v,~-{}-verbose~~~~~~~~~display~detailed~status~information

~~-s,~-{}-skip-config~~~~~skip~configure~script

~~-o,~-{}-options~~~~~~~~~options~to~pass~to~the~C~compiler/linker

\newpage{}
\end{lyxcode}

\subsection{Current Bugs and Shortcomings}

\begin{itemize}
\item Operation under Cygwin only works in verbose mode (option -v)!
\item querying installed extensions or trying to uninstall an extension
on a system that never had an extension installed will simply segfault
(missing extensions.db is not handled gracefully)
\item Installation from an internet source currently quits with {}``shell-init:
error retrieving current directory: getcwd: cannot access parent directories:
No such file or directory'' (may leave some temp files around?)
\item Binary installation mode is not exhaustively tested.
\item The installation of menu items for the old \emph{d.m} GUI is very
sensitive to the precise syntax of \emph{menu.tcl}. If the user has
altered this file substantially it is likely that registering the
extension submenu entries will fail.
\item There is no clean way to upgrade an existing extension and respecting
dependencies \emph{automatically}. You can use option {}``-f'' or
{}``--force'' to force over-writing of an extension, but it is safer
to uninstall first, then install the new version. 
\item GEM can only manage system-wide installations of extension. There
is no support for location or mapset-wide management.
\item path to GRASS dynamic libraries needs to be present in system-wider
linker paths.
\end{itemize}

\end{document}
