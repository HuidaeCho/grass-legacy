\documentstyle{article}
\title{Complete Spatial Randomness and Quadrat Methods\\ 
{\large GRASS Tutorial on {\tt s.qcount}}}
\author{James Darrell McCauley}
\date{08 January 1993}
\newcommand{\RR}{\mbox{\protect\makebox[.15em][l]{I}R}}
\begin{document}
\bibliographystyle{plain}
\maketitle
Cressie~\cite{cressie91} defines the concept of {\em
complete spatial randomness\/} (csr) as synonymous
with a {\em homogeneous\/} Poisson process in
$\RR^d\/$ (here the concern is $d=2$). In layman's
terms, the definition states that events are equally
likely to occur anywhere within an area
$A\subset\RR^d$.

\begin{figure}[h]
\begin{center}
% GNUPLOT: LaTeX picture using EEPIC macros
\setlength{\unitlength}{0.240900pt}
%\begin{picture}(674,526)(0,0)
\begin{picture}(610,503)(88,68)
\tenrm
\thicklines \path(176,68)(610,68)(610,503)(176,503)(176,68)
\put(467,240){\circle{12}}
\put(332,82){\circle{12}}
\put(497,355){\circle{12}}
\put(341,389){\circle{12}}
\put(361,345){\circle{12}}
\put(284,319){\circle{12}}
\put(498,333){\circle{12}}
\put(288,406){\circle{12}}
\put(602,396){\circle{12}}
\put(575,398){\circle{12}}
\put(408,272){\circle{12}}
\put(500,424){\circle{12}}
\put(468,330){\circle{12}}
\put(197,337){\circle{12}}
\put(561,189){\circle{12}}
\put(591,124){\circle{12}}
\put(272,107){\circle{12}}
\put(474,253){\circle{12}}
\put(276,107){\circle{12}}
\put(534,86){\circle{12}}
\put(453,72){\circle{12}}
\put(216,350){\circle{12}}
\put(455,503){\circle{12}}
\put(421,210){\circle{12}}
\put(607,71){\circle{12}}
\put(536,142){\circle{12}}
\put(374,89){\circle{12}}
\put(178,213){\circle{12}}
\put(228,169){\circle{12}}
\put(330,218){\circle{12}}
\put(610,417){\circle{12}}
\put(536,296){\circle{12}}
\put(285,121){\circle{12}}
\put(547,400){\circle{12}}
\put(609,218){\circle{12}}
\put(426,103){\circle{12}}
\put(318,446){\circle{12}}
\put(441,439){\circle{12}}
\put(176,361){\circle{12}}
\put(494,222){\circle{12}}
\put(339,353){\circle{12}}
\put(404,429){\circle{12}}
\put(467,446){\circle{12}}
\put(379,68){\circle{12}}
\put(479,127){\circle{12}}
\put(316,433){\circle{12}}
\put(506,327){\circle{12}}
\put(360,219){\circle{12}}
\put(563,354){\circle{12}}
\put(500,303){\circle{12}}
\end{picture}

\centerline{(a)}
% GNUPLOT: LaTeX picture using EEPIC macros
\setlength{\unitlength}{0.240900pt}
%\begin{picture}(674,526)(0,0)
\begin{picture}(610,503)(88,68)
\tenrm
\thicklines \path(176,68)(610,68)(610,503)(176,503)(176,68)
\put(188,84){\circle{12}}
\put(184,143){\circle{12}}
\put(192,201){\circle{12}}
\put(196,254){\circle{12}}
\put(180,293){\circle{12}}
\put(178,361){\circle{12}}
\put(176,420){\circle{12}}
\put(191,489){\circle{12}}
\put(238,68){\circle{12}}
\put(232,111){\circle{12}}
\put(233,164){\circle{12}}
\put(239,273){\circle{12}}
\put(233,290){\circle{12}}
\put(236,397){\circle{12}}
\put(240,433){\circle{12}}
\put(224,481){\circle{12}}
\put(305,68){\circle{12}}
\put(288,116){\circle{12}}
\put(286,210){\circle{12}}
\put(278,275){\circle{12}}
\put(298,293){\circle{12}}
\put(314,353){\circle{12}}
\put(292,426){\circle{12}}
\put(280,488){\circle{12}}
\put(351,85){\circle{12}}
\put(375,134){\circle{12}}
\put(333,172){\circle{12}}
\put(356,260){\circle{12}}
\put(333,319){\circle{12}}
\put(358,360){\circle{12}}
\put(335,461){\circle{12}}
\put(376,486){\circle{12}}
\put(420,84){\circle{12}}
\put(415,139){\circle{12}}
\put(400,202){\circle{12}}
\put(409,227){\circle{12}}
\put(401,292){\circle{12}}
\put(427,384){\circle{12}}
\put(407,426){\circle{12}}
\put(398,500){\circle{12}}
\put(472,78){\circle{12}}
\put(474,107){\circle{12}}
\put(482,210){\circle{12}}
\put(483,275){\circle{12}}
\put(485,294){\circle{12}}
\put(474,358){\circle{12}}
\put(494,445){\circle{12}}
\put(478,503){\circle{12}}
\put(515,72){\circle{12}}
\put(509,124){\circle{12}}
\put(549,176){\circle{12}}
\put(553,273){\circle{12}}
\put(538,336){\circle{12}}
\put(528,387){\circle{12}}
\put(535,432){\circle{12}}
\put(518,491){\circle{12}}
\put(592,83){\circle{12}}
\put(606,114){\circle{12}}
\put(610,198){\circle{12}}
\put(602,274){\circle{12}}
\put(593,303){\circle{12}}
\put(581,359){\circle{12}}
\put(570,427){\circle{12}}
\put(596,493){\circle{12}}
\put(316,100){\circle{12}}
\end{picture}

\centerline{(b)}
% GNUPLOT: LaTeX picture using EEPIC macros
\setlength{\unitlength}{0.240900pt}
%\begin{picture}(674,526)(0,0)
\begin{picture}(610,503)(88,68)
\tenrm
\thicklines \path(176,68)(610,68)(610,503)(176,503)(176,68)
\put(544,250){\circle{12}}
\put(601,290){\circle{12}}
\put(397,159){\circle{12}}
\put(610,121){\circle{12}}
\put(553,131){\circle{12}}
\put(559,123){\circle{12}}
\put(535,126){\circle{12}}
\put(508,128){\circle{12}}
\put(529,124){\circle{12}}
\put(493,127){\circle{12}}
\put(394,133){\circle{12}}
\put(311,136){\circle{12}}
\put(302,115){\circle{12}}
\put(251,121){\circle{12}}
\put(260,118){\circle{12}}
\put(212,116){\circle{12}}
\put(206,125){\circle{12}}
\put(317,68){\circle{12}}
\put(299,460){\circle{12}}
\put(550,283){\circle{12}}
\put(415,243){\circle{12}}
\put(281,211){\circle{12}}
\put(347,259){\circle{12}}
\put(359,277){\circle{12}}
\put(368,298){\circle{12}}
\put(433,322){\circle{12}}
\put(493,358){\circle{12}}
\put(502,363){\circle{12}}
\put(481,362){\circle{12}}
\put(502,376){\circle{12}}
\put(454,345){\circle{12}}
\put(514,384){\circle{12}}
\put(496,386){\circle{12}}
\put(466,365){\circle{12}}
\put(433,368){\circle{12}}
\put(463,376){\circle{12}}
\put(469,380){\circle{12}}
\put(436,386){\circle{12}}
\put(305,326){\circle{12}}
\put(221,330){\circle{12}}
\put(203,313){\circle{12}}
\put(194,329){\circle{12}}
\put(176,303){\circle{12}}
\put(245,450){\circle{12}}
\put(338,442){\circle{12}}
\put(499,396){\circle{12}}
\put(493,396){\circle{12}}
\put(526,389){\circle{12}}
\put(365,503){\circle{12}}
\put(287,475){\circle{12}}
\end{picture}

\centerline{(c)}
\end{center}
\caption{Realization of two-dimensional Poisson processes of 50
points on the unit square exhibiting (a)~complete
spatial randomness, (b)~regularity, and (c)~clustering.}
\label{fig:csr}
\end{figure}

\begin{figure}[h]
% load 'stat.inc
% plot [0:10] cchi(x) title "$\chi^2\/$ CDF", chi(x) title "$\chi^2\/$ PDF"
\begin{center}
% GNUPLOT: LaTeX picture using EEPIC macros
\setlength{\unitlength}{0.240900pt}
\begin{picture}(1500,900)(0,0)
\tenrm
\thicklines \path(176,68)(196,68)
\thicklines \path(1436,68)(1416,68)
\put(154,68){\makebox(0,0)[r]{0}}
\thicklines \path(176,203)(196,203)
\thicklines \path(1436,203)(1416,203)
\put(154,203){\makebox(0,0)[r]{0.2}}
\thicklines \path(176,338)(196,338)
\thicklines \path(1436,338)(1416,338)
\put(154,338){\makebox(0,0)[r]{0.4}}
\thicklines \path(176,472)(196,472)
\thicklines \path(1436,472)(1416,472)
\put(154,472){\makebox(0,0)[r]{0.6}}
\thicklines \path(176,607)(196,607)
\thicklines \path(1436,607)(1416,607)
\put(154,607){\makebox(0,0)[r]{0.8}}
\thicklines \path(176,742)(196,742)
\thicklines \path(1436,742)(1416,742)
\put(154,742){\makebox(0,0)[r]{1}}
\thicklines \path(176,877)(196,877)
\thicklines \path(1436,877)(1416,877)
\put(154,877){\makebox(0,0)[r]{1.2}}
\thicklines \path(176,68)(176,88)
\thicklines \path(176,877)(176,857)
\put(176,23){\makebox(0,0){0}}
\thicklines \path(428,68)(428,88)
\thicklines \path(428,877)(428,857)
\put(428,23){\makebox(0,0){2}}
\thicklines \path(680,68)(680,88)
\thicklines \path(680,877)(680,857)
\put(680,23){\makebox(0,0){4}}
\thicklines \path(932,68)(932,88)
\thicklines \path(932,877)(932,857)
\put(932,23){\makebox(0,0){6}}
\thicklines \path(1184,68)(1184,88)
\thicklines \path(1184,877)(1184,857)
\put(1184,23){\makebox(0,0){8}}
\thicklines \path(1436,68)(1436,88)
\thicklines \path(1436,877)(1436,857)
\put(1436,23){\makebox(0,0){10}}
\thicklines \path(176,68)(1436,68)(1436,877)(176,877)(176,68)
\put(1306,812){\makebox(0,0)[r]{$\chi^2\/$ CDF}}
\thinlines \path(1328,812)(1394,812)
\thinlines \path(176,68)(176,68)(189,236)(201,302)(214,350)(227,388)(240,420)(252,448)(265,472)(278,494)(291,513)(303,530)(316,545)(329,560)(341,572)(354,584)(367,595)(380,605)(392,614)(405,622)(418,630)(431,638)(443,644)(456,650)(469,656)(481,662)(494,667)(507,671)(520,676)(532,680)(545,684)(558,687)(571,690)(583,693)(596,696)(609,699)(621,702)(634,704)(647,706)(660,708)(672,710)(685,712)(698,714)(711,716)(723,717)(736,719)(749,720)(761,721)(774,722)(787,724)(800,725)
\thinlines \path(800,725)(812,726)(825,727)(838,727)(851,728)(863,729)(876,730)(889,730)(901,731)(914,732)(927,732)(940,733)(952,733)(965,734)(978,734)(991,735)(1003,735)(1016,736)(1029,736)(1041,736)(1054,737)(1067,737)(1080,737)(1092,737)(1105,738)(1118,738)(1131,738)(1143,738)(1156,739)(1169,739)(1181,739)(1194,739)(1207,739)(1220,739)(1232,740)(1245,740)(1258,740)(1271,740)(1283,740)(1296,740)(1309,740)(1321,740)(1334,741)(1347,741)(1360,741)(1372,741)(1385,741)(1398,741)(1411,741)(1423,741)(1436,741)
\put(1306,767){\makebox(0,0)[r]{$\chi^2\/$ PDF}}
\Thicklines \path(1328,767)(1394,767)
\Thicklines \path(189,873)(189,873)(201,609)(214,488)(227,414)(240,362)(252,323)(265,293)(278,268)(291,247)(303,229)(316,214)(329,201)(341,190)(354,180)(367,170)(380,162)(392,155)(405,148)(418,142)(431,137)(443,132)(456,127)(469,123)(481,119)(494,116)(507,113)(520,110)(532,107)(545,104)(558,102)(571,100)(583,98)(596,96)(609,94)(621,92)(634,91)(647,89)(660,88)(672,87)(685,86)(698,85)(711,84)(723,83)(736,82)(749,81)(761,80)(774,79)(787,79)(800,78)(812,78)
\Thicklines \path(812,78)(825,77)(838,76)(851,76)(863,76)(876,75)(889,75)(901,74)(914,74)(927,74)(940,73)(952,73)(965,73)(978,72)(991,72)(1003,72)(1016,72)(1029,72)(1041,71)(1054,71)(1067,71)(1080,71)(1092,71)(1105,70)(1118,70)(1131,70)(1143,70)(1156,70)(1169,70)(1181,70)(1194,70)(1207,70)(1220,69)(1232,69)(1245,69)(1258,69)(1271,69)(1283,69)(1296,69)(1309,69)(1321,69)(1334,69)(1347,69)(1360,69)(1372,69)(1385,69)(1398,69)(1411,69)(1423,69)(1436,69)
\end{picture}

\end{center}
\caption{Probability Density Function and Cumulative Distribution
Function for a $\chi^2\/$ Random Variable.}
\label{fig:chisq}
\end{figure}

There are two types departure from a csr: regularity
and clustering.  Figure~\ref{fig:csr} shows
realizations of three processes for $N\left(A\right) =
50.$ Notice in figure~\ref{fig:csr}a that this
process may {\em appear\/} to be clustered.
This is because event-to-nearest-event distances ($W$) of
a homogeneous Poisson process can be modeled
as $\chi_2^2\/$ random variables. Recall the form of
the $\chi_2^2$ probability density function
(fig.~\ref{fig:chisq}). Most of the probability
is near zero. It follows that, given a suitable statistic,
csr can be tested using Pearsons's $\chi^2$ goodness-of-fit test.

Various indices and statistics
measure departure from csr, ie., the pattern.
Tables 8.3 and 8.6 in
Cressie's book~\cite{cressie91} summarizes six {\em
quadrat count\/} indices and 17 {\em nearest
neighbor\/} test statistics for quantifying
departure from csr.   The former,  also discussed
in Ripley~\cite{ripley81}, are implemented
in {\tt s.qcount} and described later.

{\tt s.qcount}
chooses {\em n\/} circular
regions of radius {\em r\/} such that they
are completely within the bounds of the current geographic
region and no two
regions overlap.
The regions are called random
{\em quadrats}.\footnote{Cressie~\protect\cite{cressie91} 
defined random quadrats this way, however Ripley~\protect\cite{ripley81}
used rectangular quadrats that potentially overlapped.}
The number of sites falling within each 
quadrat are counted and indices are calculated to estimate 
the departure of site locations from complete spatial 
randomness. This is illustrated in figure~\ref{fig:quads}.

\begin{figure}[h]
\begin{center}
\input pinequads.latex
\end{center}
\caption{Randomly placed quadrats ($n=100$) with 584 sample points. This figure
was produced using the output of {\tt s.qcount} and {\tt g.gnuplot}.}
\label{fig:quads}
\end{figure}

The six indices and their realizations for the sampling
shown in figure~\ref{fig:quads} are shown in Table~\ref{tbl:indices}.
Cressie~\cite{cressie91} gives a short summary of these on pages~590
and 591 and Ripley~\cite{ripley81} discusses them on pages~102--106.
The original reference to each index is given in Table~\ref{tbl:indices}
in case you want to read more about them.

\begin{table}[h]
\begin{center}
\begin{minipage}{.8\textwidth}
\caption{Indices for Quadrat Count Data.  Adapted from
Cressie~\protect\cite{cressie91}, this table shows the
statistics computed for the quadrats in figure~\protect\ref{fig:quads}}
\label{tbl:indices}
\begin{center}
\begin{tabular}{lccl}\hline
Index & Estimator\protect\footnote{$X_i$ the
number of sites in the $i$th quadrat, $\bar{X}\/$ is the mean of the
quadrat counts, and $S^2\/$ is the sample variance.}
 & Realization & \multicolumn{1}{c}{Reference}\\ \hline
$I$ & $\frac{S^2}{\bar{X}}$ & 2.128 & Fisher~\cite{fisher22} \\ \\
$ICS$ & $\frac{S^2}{\bar{X}}-1$ & 1.128& David and Moore~\cite{david54}\\ \\
$ICF$ & $\frac{\bar{X}^2}{S^2-\bar{X}}$ & 1.383& Douglas~\cite{douglas75}\\ \\
$\stackrel{*}{X}$ & $\bar{X}-\frac{S^2}{\bar{X}}-1$
				& 2.688& Lloyd~\cite{lloyd67}\\ \\
$IP$ & $\frac{\stackrel{*}{X}}{\bar{X}}$ 
				& 1.723 & Lloyd~\cite{lloyd67}\\ \\
$I_\delta$ & 
$\frac{n\sum_{i=1}^n X_i\left(X_i-1\right)}{n\bar{X} \left(n\bar{X}-1\right)}$ 
%$\frac{n\bar{X} \cdot IP}{n\bar{X}-1}$ 
& 1.720 & Morisita~\cite{morisita59}\\ \hline
\end{tabular}
\end{center}
\end{minipage}
\end{center}
\end{table}

\clearpage
\bibliography{quadrat}
\end{document}
